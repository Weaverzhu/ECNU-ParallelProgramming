
\begin{frame}
    \frametitle{solving matrix mult on CUDA}

    \begin{itemize}
        \item host vs device, distributed memory
        \item streamprocessors
        \item blocks, wraps, threads, shared memory
    \end{itemize}

    子任务:给定 $i,j$, 计算 $C_{ij}$,也就是并行最外面的两个for
\end{frame}

\begin{frame}
    \frametitle{solving matrix mult on CUDA}

    \begin{itemize}
        \item 轻量级线程,切换成本较低
        \item 基本可以平等划分任务
        \item 一个线程对应一个子任务(开 $n \times m$ 个线程)
    \end{itemize}
\end{frame}

\begin{frame}
    \frametitle{shared memory}
    \begin{itemize}
        \item 公式: $C_{ij} = \sum\limits_{k=0}^{bm} A_{ik} \cdot B_{kj}$
        \item 对于内层的两个循环$j,k$,发现公用 $A_{ix}$
        \item 使用 share memory 缓存 $A$
    \end{itemize}
\end{frame}

\begin{frame}
    \frametitle{share memory}
    \begin{itemize}
        \item block内公用
        \item \texttt{block\_size} 必须是 $am$ 的因子,保证同块内使用相同的一行
        \item 别忘了 \texttt{\_\_sychronize}
    \end{itemize}
\end{frame}

\begin{frame}
    \frametitle{Fast IO: buffered input \& output}
    \begin{itemize}
        \item \texttt{fread, fwrite instead of printf, scanf}
        \item no need to flush every time we read/write a number
    \end{itemize}
\end{frame}

\begin{frame}
    \frametitle{Occupancy: optimal size for the number of blocks and threads}
    \begin{itemize}
        \item 以 wrap 为计算单元
        \item thread 中有register
        \item block 中有share memory
        \item 每个 multiprocessor 也有最大的 register 和 share memory 和  active threads
        \item 如果资源不够将无法派出所有的资源进行运算
    \end{itemize}
\end{frame}

\begin{frame}
    \frametitle{Occupancy: optimal size for the number of blocks and threads}
    \begin{itemize}
        \item 我们需要让最大比例的显卡工作
        \item 引入 Occupancy 机制
        \item 通过 nvidia 官网上的 Occupancy calculator 进行运算
        \item \texttt{nvcc --ptxas-options=-v to nvcc} 查看内核函数使用寄存器个数
        \item 使用 \texttt{}
    \end{itemize}
\end{frame}

\begin{frame}
    \frametitle{Bottleneck: IO}
    \begin{itemize}
        \item GPU 计算矩阵瓶颈在于 IO
        \item 输出浮点数也会带来计算
    \end{itemize}
\end{frame}

\begin{frame}
    \frametitle{Bottleneck: IO}
    \begin{itemize}
        \item 仿照 MPI 在 distributed memory 中使用异步 IO
        \item Host 一边传入 Device 的数据一边输出到文件
        \item Host 一边读入文件一遍 Copy 数据到 Device
    \end{itemize}
\end{frame}

\begin{frame}
    \frametitle{asychronize IO: input}
    \begin{itemize}
        \item \texttt{cudaStream_t}
        \item \texttt{cudaHostAlloc} instead of \texttt{malloc}
        \item \texttt{cudaMemcpyAsync} instead of \texttt{cudaMemcpy}
    \end{itemize}
\end{frame}

\begin{frame}
    \frametitle{solving matrix mult on CUDA}

    \begin{itemize}
        \item host vs device, distributed memory
        \item streamprocessors
        \item blocks, wraps, threads, shared memory
    \end{itemize}

    子任务:给定 $i,j$, 计算 $C_{ij}$,也就是并行最外面的两个for
\end{frame}

\begin{frame}
    \frametitle{solving matrix mult on CUDA}

    \begin{itemize}
        \item 轻量级线程,切换成本较低
        \item 基本可以平等划分任务
        \item 一个线程对应一个子任务(开 $n \times m$ 个线程)
    \end{itemize}
\end{frame}

\begin{frame}
    \frametitle{shared memory}
    \begin{itemize}
        \item 公式: $C_{ij} = \sum\limits_{k=0}^{bm} A_{ik} \cdot B_{kj}$
        \item 对于内层的两个循环$j,k$,发现公用 $A_{ix}$
        \item 使用 share memory 缓存 $A$
    \end{itemize}
\end{frame}

\begin{frame}
    \frametitle{share memory}
    \begin{itemize}
        \item block内公用
        \item \texttt{block\_size} 必须是 $am$ 的因子,保证同块内使用相同的一行
        \item 别忘了 \texttt{\_\_sychronize}
    \end{itemize}
\end{frame}

\begin{frame}
    \frametitle{Fast IO: buffered input \& output}
    \begin{itemize}
        \item \texttt{fread, fwrite instead of printf, scanf}
        \item no need to flush every time we read/write a number
    \end{itemize}
\end{frame}

\begin{frame}
    \frametitle{Occupancy: optimal size for the number of blocks and threads}
    \begin{itemize}
        \item 以 wrap 为计算单元
        \item thread 中有register
        \item block 中有share memory
        \item 每个 multiprocessor 也有最大的 register 和 share memory 和  active threads
        \item 如果资源不够将无法派出所有的资源进行运算
    \end{itemize}
\end{frame}

\begin{frame}
    \frametitle{Occupancy: optimal size for the number of blocks and threads}
    \begin{itemize}
        \item 我们需要让最大比例的显卡工作
        \item 引入 Occupancy 机制
        \item 通过 nvidia 官网上的 Occupancy calculator 进行运算
        \item \texttt{nvcc --ptxas-options=-v to nvcc} 查看内核函数使用寄存器个数
        \item 使用 \texttt{}
    \end{itemize}
\end{frame}

\begin{frame}
    \frametitle{Bottleneck: IO}
    \begin{itemize}
        \item GPU 计算矩阵瓶颈在于 IO
        \item 输出浮点数也会带来计算
    \end{itemize}
\end{frame}

\begin{frame}
    \frametitle{Bottleneck: IO}
    \begin{itemize}
        \item 仿照 MPI 在 distributed memory 中使用异步 IO
        \item Host 一边传入 Device 的数据一边输出到文件
        \item Host 一边读入文件一遍 Copy 数据到 Device
    \end{itemize}
\end{frame}

\begin{frame}
    \frametitle{asychronize IO: input}
    \begin{itemize}
        \item \texttt{cudaStream\_t}
        \item \texttt{cudaHostAlloc} instead of \texttt{malloc}
        \item \texttt{cudaMemcpyAsync} instead of \texttt{cudaMemcpy}
    \end{itemize}
\end{frame}

% fastio
\begin{frame}[fragile]
    \frametitle{code: buffered input}
    \begin{cppcode}
    inline char next_char()
    {
        static char buf[100000], *p1 = buf,
            *p2 = buf;
        return p1 == p2 && (p2 = (p1 = buf)
            + fread(buf, 1, 100000, fin), p1 == p2) 
            ? EOF : *p1++;
    }
    \end{cppcode}
\end{frame}


\begin{frame}[fragile]
    \frametitle{code: buffered output}
    \begin{cppcode}
    inline void flush() {
        fwrite(buffer, 1, s-buffer, stdout);
        s = buffer;
        fflush(stdout);
    }
    inline void print(const char ch) {
        // putchar(ch); return;
        if (s-buffer>OutputBufferSize-2) flush();
        *s++ = ch;
    }
    \end{cppcode}
\end{frame}
 
 

\begin{frame}[fragile]
    \frametitle{code: share memory, block partition}
    \begin{cppcode}
    __shared__ ld c_a[max_shared_size];
    int index = blockDim.x * blockIdx.x + threadIdx.x;
    if (index >= an * bm) return;
    int st = min(index, addi) * (workload+1) + 
    max(0, index - addi) * workload, ed = 
    st + workload + (index < addi ? 1 : 0);
    int shareda = min(am, max_shared_size);
    for (int p=st; p<ed; ++p) {
        // ...
    \end{cppcode}
\end{frame} 

\begin{frame}[fragile]
    \frametitle{code: share memory, block partition}
    \begin{cppcode}
    for (int p=st; p<ed; ++p) {
        int i = p / bm, j = p % bm;
        if (p % bm == 0) {
            
            for (int j=0; j<shareda; ++j) {
                c_a[j] = d_a[i * am + j];
            }
            __syncthreads();
        }
        ...
    \end{cppcode}
\end{frame}

\begin{frame}[fragile]
    \frametitle{code: share memory, block partition}
    \begin{cppcode}
    for (int p=st; p<ed; ++p) {
        int i = p / bm, j = p % bm;
        if (p % bm == 0) {
            
            for (int j=0; j<shareda; ++j) {
                c_a[j] = d_a[i * am + j];
            }
            __syncthreads();
        }
        ...
    \end{cppcode}
\end{frame}


\begin{frame}[fragile]
    \frametitle{code: async IO, creating streams}
    \begin{cppcode}
    int st = 0, ed = n * m;
    // printf("st=%d ed=%d, a=%p\n", st, ed, a); 
    cudaStream_t stream[2];
    int mask = 0;
    cudaStreamCreate(&stream[0]);
    cudaStreamCreate(&stream[1]);
    int size;
    \end{cppcode}
\end{frame}

\begin{frame}[fragile]
    \frametitle{code: async IO, creating streams}
    \begin{cppcode}
    cudaStream_t mainstream;
    cudaStreamCreate(&mainstream);
    // ...
    copyMatrixAsync(h_a, d_a, an, am, mainstream);
    // ... read h_b
    copyMatrixAsync(h_b, d_b, bn, bm, mainstream);
    // ... 
    matrixMult<<<grids, block_size, 0, mainstream>>>
    (d_a, d_b, d_c, an, bm, am);
    // ...
    handleCudaError(cudaStreamSynchronize(mainstream));
    // ...
    outputMatrixAsync(h_c, d_c, n, m);
    // pipeline: memcpy from device & output each rows 
    \end{cppcode}
\end{frame}

\begin{frame}[fragile]
    \frametitle{code: async IO, pipelines}
    \begin{cppcode}
    for (; st<ed; st+=size, mask^=1) {
        size = min(chunk_size, ed - st);
        handleCudaError(cudaMemcpyAsync(a + st,
         d_a + st, size * sizeof(ld), 
         cudaMemcpyDeviceToHost, stream[mask]));
        // exit(0);
        if (st - chunk_size >= 0) {
            // printf("%d %d\n",st-chunk_size, st);
            handleCudaError(cudaStreamSynchronize(stream[mask^1]));
            outputinterval(a, st-chunk_size, st);
        }
    }
    \end{cppcode}
\end{frame}

